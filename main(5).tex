\documentclass[a4paper]{article}
\usepackage[utf8]{inputenc}
\usepackage{graphicx}
\usepackage{amsmath}
\usepackage[bottom=2.0cm,top=2.0cm,left=2.0cm,right=2.0cm]{geometry}
\usepackage[portuges]{babel}
\usepackage{natbib}
\usepackage{graphicx}
\usepackage{indentfirst}

\begin{document}

\begin{titlepage}
	\begin{center}		
	\begin{figure}[htb!]
			\begin{flushleft}
			\includegraphics[width=3.9cm]{logo_unb}
			\end{flushleft}
\end{figure}
        \vspace{-2.5cm}
        \hspace{2.1cm}\Large{\textbf{Universidade de Brasília}}\\
        \hspace{2.1cm}\Large{Cálculo 3-1/2020}\\
        \vspace{200pt}
        \LARGE{\textbf{Trabalho:}}\\ 
        \Large{Contaminação Epidêmica}\\
        \vspace{150pt}
        \vspace{40pt}
        \hfill Gabriella Magalhães\hspace{20pt}18/0120964\\
        \hfill Pedro Derze\hspace{20pt}17/0163814  \\
        \hfill Eduardo Ferreira \hspace{20pt}19/0026987\\
        \hfill Felipe Goulart \hspace{20pt}15/0153651 
       \vspace{\fill}

	\end{center}
\end{titlepage}
\newpage
\LARGE{\section{Introdução}}

Nossa proposta é estudar a contaminação epidêmica de uma certa região.\newline

Suponhamos que o local de interesse seja a região circular com um raio de $10Km$ onde, no centro, há o Hospital Regional da Asa Norte. Consideremos que os indivíduos que habitam essa região se distribuam uniformemente por ela e que a taxa de contaminação por certa doença seja de $k$ pessoas por $Km^2$. Consideremos ainda a função probabilidade, dada por\newline

\begin{equation}
    f(P)= \int\int_D \frac{k}{20}[20-d(P,A)]dxdy ,
\end{equation}

onde $d(P,A)$ é a distância entre $P$ e $A$, sendo $A$ um indivíduo não infectado no ponto $A(x_0,y_0)$. Neste caso, essa função mostra a probabilidade de um indivíduo ser infectado.\newline

Nosso questionamento é: uma pessoa que mora no centro tem a mesma chance de contaminação que uma pessoa que mora na periferia? Ora! Agora que temos as ferramentas para essa resposta, vamos checar!\newpage

\section{}
Para uma pessoa que mora no centro, sua posição $A$ é $A=(0,0)$ e a função probabilidade é dada por\newline

\begin{equation}
    f(x,y)= \int\int_D \frac{k}{20}[20-\sqrt{x^2+y^2}]dxdy ,
\end{equation}
em que $D$ é o domínio do interior da área circular:\newline

$D:\{ (x,y)\in \mathcal{R}^2 ; x^2+y^2\leq 10^2 \}$.\newline

Essa integral fica mais simples se utilizarmos coordenadas polares. Adaptando o domínio para $\widehat{D}$, dado por\newline

$\widehat{D}:\{ 0\leq r\leq 10; 0\leq \theta \leq 2\pi \}$,\newline

temos que a integral $(2)$ fica como

\begin{equation}
    f(r,\theta)=\int_0^{2\pi}\int_0^{10}\frac{k}{20}[20-r]rdrd\theta,
\end{equation}\newline
que agora sim é uma integral sem grandes complicações.\newline
Resolvendo-a, então, obtemos\newline

$
    f(r,\theta)=\frac{k}{20}2\pi\int_0^{10}(20r-r^2)dr = \frac{k\pi}{10}[10r^2-\frac{r^3}{3}]/_0^{10}=\frac{200k\pi}{3}\approx 209k
$\newline

Já para uma pessoa que mora na periferia, sua posição é dada por um ponto na circunferência, $A=(10,0)$, por exemplo. A função probabilidade seria, então, \newline

$ f(x,y)= \int\int_D \frac{k}{20}[20-\sqrt{(x-10)^2+y^2}]dxdy$.\newline

Essa integral, entretanto, é complicada de resolver.\newline
Uma maneira conveninente de torná-la mais simples é transladar os eixos para que a origem seja em $A$ e escrevê-la em coordenadas polares, como na figura abaixo.

\begin{figure}[htb!]
    \centering
    \includegraphics[width=15cm]{coordpol.png}
\end{figure}

O domínio $\widehat{D}$, então, seria $\widehat{D}:\{ \frac{-\pi}{2}\leq \theta \leq \frac{\pi}{2}; 0\leq r\leq 20cos\theta \}$. Assim, a função probabilidade fica\newline

\begin{equation}
    f(r,\theta)= \int_{\frac{-\pi}{2}}^{\frac{\pi}{2}}\int_0^{20cos\theta}\frac{k}{20}[20-r]rdrd\theta
\end{equation}\newline
Resolvendo, por fim:\newline

$
    k\int_{\frac{-\pi}{2}}^{\frac{\pi}{2}}\int_0^{20cos\theta}[1-r/20]rdrd\theta=
    k\int_{\frac{-\pi}{2}}^{\frac{\pi}{2}}[\frac{r^2}{2}-\frac{r^3}{60}]/_0^{20cos\theta}d\theta=\newline
    k\int_{\frac{-\pi}{2}}^{\frac{\pi}{2}}(200cos^2\theta-\frac{400}{3}cos^3\theta)d\theta=200k(\frac{\theta}{2}+\frac{sen(2\theta)}{4}-\frac{2sen\theta}{3}+\frac{2}{3}.\frac{1}{3}sen^3\theta)/_{\frac{-\pi}{2}}^{\frac{\pi}{2}}=\newline
    200k(\frac{\pi}{2}-\frac{8}{9})\approx136k.
$\newline

Dessa forma, podemos concluir que a probabilidade de contaminação em áreas periféricas é menor que em áreas centrais.\newpage

Esse resultado se confirma quando analisamos dados da Covid-19 no Distrito Federal. A tabela abaixo mostra a incidência da doença a cada $100$ mil habitantes no mês de abril deste ano.\newline

\begin{figure}[htb!]
    \centering
    \includegraphics[width=15cm]{covid.jpeg}
\end{figure}

 \footnote{Reportagem completa em: https://glo.bo/34QTG6y}\newpage

\section{Bibliografia}
[1] Stewart, James. Cálculo 2, 6ª ed. Norte-Americana, Cencage Learning, São Paulo 2011; Página 940, exercício 33.

\end{document}
