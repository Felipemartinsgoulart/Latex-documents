\documentclass[12pt, a4paper]{article}
\usepackage[utf8]{inputenc}
\usepackage{graphicx}
\usepackage{amsmath}
\usepackage[bottom=2.0cm,top=2.0cm,left=2.0cm,right=2.0cm]{geometry}
\usepackage[Portuges]{babel}
\usepackage{multicol}
\usepackage{indentfirst}
\usepackage{hyperref} 
\usepackage{longtable}
\usepackage{gensymb}
\usepackage{setspace}
\hypersetup{colorlinks,citecolor=black,filecolor=black,linkcolor=black,urlcolor=black} 

\title{Universidade de Brasília\\
Instituto de Física}
\author{}
\date{May 2019}

\begin{document}

\maketitle

\begin{center}
Fisica Geral 4 Experimental \\ Turma A\\ Experimento 5 - O Caráter Quântico da Luz e a Constante de Planck
\end{center}
\begin{flushleft}

\end{flushleft}

\section{Objetivo}
\quad Verificar a validade das hipóteses quântica e clássica da luz estudando o efeito fotoelétrico. Determinar a constante de Planck e a função trabalho do fotodiodo.

\section{Introdução Teórica}
\quad A lei da radiação de Planck, publicada em 1901, afirma que um oscilador ou qualquer sistema físico semelhante, possui um conjunto discreto de níveis de energia possíveis e que não poderia existir nenhum valor de energia entre esses valores. A lei também afirmava que a emissão ou absorção de radiação estava associada a transições ou saltos entre dois níveis de energia. Essa troca de energia feita pelo oscilador é medida em um quanta de energia radiante e é possível calcular sua magnitude através da expressão:
\[E=h\cdot v \eqno(1)\]
\quad Ou seja a energia radiante (E) é igual a uma constante da natureza(h) vezes a frequência da radiação, a constante da natureza (h) ficou logo conhecida e foi batizada como constante de Planck. O novo modelo proposto traz uma nova abordagem para a emissão fotoelétrica, nesse modelo  diz que luz de frequências mais altas produziria elétrons mais energéticos, não importando a intensidade da mesma, que só mudaria com o número de elétrons emitidos por segundo. Einstein utilizou a teoria de Planck para explicar o efeito fotoelétrico em termos da teoria quântica utilizando a expressão:

\[\frac{1}{2}m\cdot V^{2}=h\cdot v-W_{0} \eqno(2)\]
\quad Onde é possível igualar a energia cinética dos fotoelétrons emitidos com a energia radiante, calculada por Planck, menos o trabalho necessário para remover os elétrons da superfície do material. Einstein recebeu o Nobel em 1921 pelos seus estudos. 

\quad Experimentalmente se um fóton com energia hv incide sobre um catodo em um tubo evacuado ele consegue arrancar um elétron da superfície desde que a energia seja maior que a função trabalho do material. O elétron vai utilizar parte da enrgia adquirida para escapar do catodo e o resto da anergia na forma de energia cinética. Os elétrons emitidos conseguem chegar ao anodo do tubo e podem ser detectaodos pois formam uma corrente elétrica. Se quisermos anular a corrente é necessária igualar a energia cinética da potencial necessária para zerar a corrente obtemos a expressão:

 \[\frac{1}{2} m\cdot v^{2}=e\cdot V\]
 
\quad Substituindo pela equação (2) temos:

 \[e\cdot V=h\cdot v-W_{0}\]
 
\quad A partir daí é possível obter o potencial de parada dos elétrons com a frequência do fóton pela expressão:

\[V=\frac{h}{e}v-\frac{W_{0}}{e} \eqno(3)\]

\quad A constante de Planck aparece após perceber que no gráfico da função acima a inclunação da reta é h/e onde o corte com a ordenada se dá em $-W_{0}/e$, sabendo $e = 1,602\times 10^{19}$ coulombs, é possível determinar a constante de Planck experimentalmente.


\section{Procedimentos}
\quad \textbf{3.1} Dependência da energia dos fotoelétrons com a intensidade luminosa.\\

 \quad Com o intuito de analisar o efeito que a intensidade da luz sobre o fotodiodo, posicionamos o equipamento de tal modo que somente uma linha espectral incidisse sobre o fotodiodo. Pressionamos o botão da lateral do módulo para certificarmos que o potencial registrado pelo voltímetro era devido, somente, à incidência da linha de emissão escolhida. Utilizando um cronômetro, registramos quanto tempo era necessário para que a leitura do potencial se estabilizasse e o valor da voltagem máxima. Começamos deixando passar a luz com 100\% da intensidade, depois repetimos o procedimento usando o filtro de transmissão variável para diminuir a intensidade da luz que incidia sobre o fotodiodo. O procedimento foi feito para as 5 linhas espectrais da lâmpada de mercúrio. Para as linhas de emissão verde e amarela usamos um filtro correspondente para tentar eliminar a influência de outros comprimentos de onda no fotodiodo. Os dados estão registrados na tabela 1.\\
 
 \quad \textbf{3.2} Dependência da energia dos fotoelétrons com a frequência da luz incidente\\
 
 \quad Registramos o valor máximo de potencial de parada de cada linha de emissão da lâmpada de mercúrio e relacionamos com os valores de frequência correspondente. Os valores\ de frequência foram fornecidos pelo manual que acompanho o kit do experimento. O dados foram registrados na tabela 6.\\
 
 \quad \textbf{3.3} Relação entre energia e frequência\\
 
\quad Para determinarmos a relação entre a energia e frequência, plotamos o gráfico com os os dados obtidos no procedimento anterior. 

\section{Materiais}
\begin{itemize}
    \item Voltímetro digital
    \item Módulo h/e (AP-9368) da PASCO
    \item Kit de acessórios para o módulo h/e (AP-9368) da PASCO
    \item Fonte da luz de vapor de mercúrio (OS-9286) da PASCO
\end{itemize}

\section{Dados Experimentais e Análise de Dados}
\quad \textbf{5.1}  Nas tabelas abaixo estão registrados os dados referentes ao potencial de parada e tempo aproximado de recarga em função da intensidade da luz transmitida para cada cor.\\

\begin{center}

\quad Tabela 1 - Potencial de parada e tempo aproximado de recarga em função da intensidade de luz trasmitida.\\

    \begin{tabular}{|c|c|c|c|} \hline
        Cor & \% transmissão & Potencial de parada (V) & Tempo aprox. de carga (s) \\ \hline
         & 100 & 1,95 &  8,16 \\  \cline{2-4} 
       UVA & 80 & 1,94 & 8,64 \\ \cline{2-4} 
        & 65 & 1,93 & 9,68 \\ \cline{2-4} 
        & 35 & 1,93 & 14,51 \\ \hline
         & 100 & 1,77 &  8,16 \\  \cline{2-4} 
       Violeta & 80 & 1,76 & 8,64 \\ \cline{2-4} 
        & 65 & 1,75 & 9,68 \\ \cline{2-4} 
        & 35 & 1,70 & 14,51 \\ \hline
        & 100 & 1,58 &  8,16 \\  \cline{2-4} 
       Azul & 80 & 1,58 & 8,64 \\ \cline{2-4} 
        & 65 & 1,57 & 9,68 \\ \cline{2-4} 
        & 35 & 1,54 & 14,51 \\ \hline
         & 100 & 0,93 &  8,16 \\  \cline{2-4} 
       Verde & 80 & 0,93 & 8,64 \\ \cline{2-4} 
        & 65 & 0,92 & 9,68 \\ \cline{2-4} 
        & 35 & 0,89 & 14,51 \\ \hline
         & 100 & 0,79 &  8,16 \\  \cline{2-4} 
       Amarelo & 80 & 0,80 & 8,64 \\ \cline{2-4} 
        & 65 & 0,80 & 9,68 \\ \cline{2-4} 
        & 35 & 0,77 & 14,51 \\ \hline
        \end{tabular}\\

\end{center}
\newpage
\quad Usando os dados das tabelas acima plotamos o gráfico 1

\begin{center}
    \begin{figure}[h]
        \centering
        \includegraphics{G1.jpg}
        \label{fig:G1}
    \end{figure}
    Gráfico 1: potencial de parada em função da intensidade.
\end{center}
\\
\quad Assim, podemos perceber que a potência de parada não depende da intensidade que a luz incide no fotocatodo. 

\quad Com a análise do podemos perceber que o potencial de parada diminui com o passar do tempo, essa queda é devido ao efeito joule, que faz com que a resistência aumente e consequentemente passe menos corrente no voltímetro. \\

\quad \textbf{5.2} Podemos perceber que cada cor vai ter um potencial de parada de acordo com sua frequência.

\begin{center}
\quad Tabela 2 - Potencia de parada em função da frequância da luz incidente.
    \begin{tabular}{|c|c|c|}\hline
        Cor da luz & Frequência (Hz) & Potencial de parada (V) \\ \hline
         Amarelo & $5,18672\times 10^{14}$ & 0,79\\ \hline
         Verde & $5,48996\times 10^{14}$ & 0,93\\ \hline
         Azul & $6,87858\times 10^{14}$ & 1,58\\ \hline
         Violeta & $7,40858\times 10^{14}$ & 1,77\\ \hline
         Ultravioleta & $8,20264\times 10^{14}$ & 1,95 \\ \hline
    \end{tabular}
\end{center}

\quad Analisando a tabela 2 vemos que o potencial maior é correspondente a maior frequência, consequentemente a cor de maior frequência vai ter uma energia máxima dos elétrons superior a energia máxima das cores de frequências mais baixas.

\quad Com os procedimentos anteriores, podemos confirmar a teoria, já que ela nos diz que a energia não depende da intensidade da luz que incidente e sim da frequência da luz, essas afirmações foram testadas  e comprovadas nas etapas anteriores.\\

\newpage
\quad \textbf{5.3} Analisando a tabela 2 plotamos do potencial de parada em função da frequência.

\begin{center}
    \begin{figure}[h]
        \centering
        \includegraphics{G2.jpg}
        \label{fig:my_label}
    \end{figure}
    Gráfico 2: potencial de parada em função da frequência.
\end{center}

\quad E a partir da análise do gráfico podemos definir tanto a constante de Planck ($h$) quanto a função trabalho ($W_{0}$)

\quad Utilizando a equação (3)

\[V=\frac{h}{e}v-\frac{W_{0}}{e}\]
onde $h/e$ é o coeficiênte angular que corresponde a $a=4,006\cdot 10^{-15} \pm 2,981\cdot 10^{-16}$, assim multiplicando esse valor pelo valor de $e=1,602\cdot 10^{-19}$ encontramos o valor para o valor da constante $h$

\[h=e\cdot a \]
\[h=1,602\cdot 10^{-19} \cdot 4,006\cdot 10^{-15}\]
\[h=6,428\cdot 10^{-34} J\cdot s\]
e o erro de $h$ é
\[\Delta h= 4,776 \cdot 10^{-35}\]

\quad A função trabalho vai ser dada por menos o coeficiente linear obtido na regressão do gráfico, que equivale a $-b=1,253\pm 2,006\cdot 10^{-1}eV$. Assim temos 

\[h=6,428\cdot 10^{-34}\pm  4,776 \cdot 10^{-35} J\cdot s\]
e
\[W_{0}=1,253\pm 2,006\cdot 10^{-1} eV\]

\section{Conclusão}
\quad O modelo de Einstein é testado e comprovado com os dados obtidos no experimento, nos primeiros procedimentos verificamos que a intensidade luminosa não tem relação com a emissão de elétrons mais energéticos, isso fica claro quando o potencial de parada teve uma variação ínfima permanecendo praticamente constante, já a corrente de elétrons estabelece uma relação proporcional, sendo maior o tempo para o potencial de parada a medida que intensidade que chegava ao diodo diminuía. No segundo procedimento a relação do potencial de parada com a frequência da luz é testada e fica claro que os dois possuem uma relação linear por quanto maior a frequência da cor utilizada maior o potencial necessário para frear os elétrons emitidos.
Após os dois experimentos fica bem claro como o modelo quântico é assertivo em relação a luz. Por último calculamos a função trabalho do fotodiodo e a constante de Planck, e pelos valores próximos aos esperados certificamos que o experimento entregou resultados consideravelmente precisos. 
\\

\section{Referências Bibliográficas}

$^{1}$ Física 4, $4^{a}$ edição. D. Halliday, R. Resnick e K. S. Krrane, LTC (Rio de Janeiro, 1996); Capítulo 49 A luz e a física quântica.

$^{2}$ Fundamentos da Física 4, Ótica e Física Moderna, $4^{a}$ edição, D. Halliday, R. Resnick e J. Walker, LTC (Rio de Janeiro, 1995). Capítulo 43 Física quântica I.
\end{document}
