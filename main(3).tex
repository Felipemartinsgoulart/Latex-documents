\documentclass[a4paper]{article}
\usepackage[utf8]{inputenc}
\usepackage{amsmath}
\usepackage[bottom=2.0cm,top=2.0cm,left=2.0cm,right=2.0cm]{geometry}


\title{%
  Estrutura da Matéria - Física Atômica \\
  \large Instituto de Física - Universidade de Brasília}

\author{Felipe Martins Goulart - 15/0153651}
\date{Fevereiro 2021}

\begin{document}


\maketitle
 \renewcommand{\theenumi}{\alph{enumi}}
\section {Questão 2}
 \begin{enumerate}
    \item
        Obtenha a frequência para a radiação absorvida que provoca uma transição entre o nível $n=3$ e o nível $n=5$.
            \begin{itemize}
                Pelo quarto postulado sabemos que as transições eletrônicas são descritas pela equação
                    \begin{equation}
                        {\Delta{E} = E_2 - E_1}\\
                    \end{equation}
                E que ao passar de um estado para o outro o elétron absorve ou emite um quantum de energia $hv$, que é igual a diferença entre os estados, portanto
                    \begin{equation}
                        {\Delta{E} = hv}\\
                    \end{equation}
                A energia em um estado estacionário é dada por: 
                    \begin{equation}
                        {{E} = -A\frac{Z^2}{n^2} = \frac{-13,605}{n^2} eV}
                    \end{equation}
                Utilizando as equações 1 e 3 teremos:
                    \begin{equation}
                        \begin{aligned}
                            &\Delta{E} =  E_2 - E_1\\
                            &\Delta{E} =  \frac{-13,605}{3^2} - (\frac{-13,605}{5^2}) \\
                            &\Delta{E} =  -1,51 + 0,599\\
                            &\Delta{E} =  0,966 eV\\
                        \end{aligned}
                    \end{equation}
                Utilizando a equação 2, temos:
                    \begin{equation}
                        \begin{aligned}
                             {\Delta{E} = hv = hv = 0,966 eV}\\
                             {v = \frac{0,966eV}{h} = \frac{0,966}{4,14.10^{-13}}}\\
                             {v = 2,33.10^{14}Hz}\\
                        \end{aligned}     
                    \end{equation}

            \end{itemize}
    \item
        A que região do espectro eletromagnético essa radiação pertence?
            \begin{itemize}
                A frequência está na ordem de $10^{14} Hz$, fica muito próximo do ultravioleta mas ainda é abaixo de 4,3 portanto a frequência está na região do infravermelho.
            \end{itemize}
\end{enumerate}



\end{document}
