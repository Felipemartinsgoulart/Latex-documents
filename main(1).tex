\documentclass{article}
\usepackage[utf8]{inputenc}

\title{Física Moderna Prova 1}
\author{Felipe Martins Goulart - 150153651}
\date{Outubro 2020}

\begin{document}

\maketitle

\section {Questão 1}
\begin{equation}
    { \frac{P_s}{4 \pi R^2} = \sigma T_r^4}
\end{equation}
\begin{equation}
    { \frac{\theta}{2} = \frac{r}{R}}
\end{equation}
\begin{equation}
    {T_s^4 = T_t^4 . 4(\frac{R}{r}^2 =T_t^4 . 4(\frac{R}{\frac{\theta{R}}{2}^2}) = \frac{T_t^4 .4.4}{\theta^2} \approx 6.115 K}
\end{equation}
\section {Questão 2}
Pela densidade de energia de Plank, temos:
\begin{equation}
    { P_t(v)dv = \frac{8 \pi v^2}{c^3} = \frac{hv}{e^{\frac{hv}{kt}} - 1} =  \frac{8 \pi hv^3}{c^3} = \frac{1}{e^{\frac{hv}{kt}} - 1}}
\end{equation}
Para um comprimento de onda maior a frequência deve ser menor portanto:

\begin{equation}
    { \frac{1}{e^{\frac{hv}{kt}} - 1} = \frac{1}{ 1 + (\frac{hv}{kt})+(\frac{hv}{kt}^3) + ... + (\frac{hc}{kt}^n) - 1} \approx \frac{1}{\frac{hv}{kt}} = \frac{kt}{hv}}  
\end{equation}
Substituindo 4 em 5, temos:

\begin{equation}
    {P_t(v)dv = \frac{8 \pi v^2}{c^3}KT} .
\end{equation}
Que é exatamente a lei de Rayleight-Jeans para baixas frquências.

\section {Questão 3}
\begin{equation}
    {I =\frac{P}{A} = 12 \frac{w}{m^2}}
\end{equation}
\begin{equation}
    {P =IA = 48 \frac{J}{s}}
\end{equation}
Pela energia, temos:
\begin{equation}
    {E = hv}
\end{equation}
para a velocidade da luz, sabemos que:
\begin{equation}
    {c = \lambda v =}
\end{equation}
De onde tiramos:
\begin{equation}
    {v = \frac{c}{\lambda}}
\end{equation}
Utilizando 11 em 9, temos:
\begin{equation}
    {E = \frac{hc}{\lambda} = 3,31.10^{-19}J}
\end{equation}
Para saber a quantidade de fótons que atingem a superfície por segundo basta dividir P por E, portanto:
\begin{equation}
    {\frac{P}{E} = \frac{48 \frac{J}{s}}{3,31.10^{-19}s} = 1,45.10^{-20}\frac{fótons}{segundo}}
\end{equation}


\section {Questão 4}
Calcule a variação $\frac{\Delta E}{E}$ na energia de um fóton que resulta no espalhamento Compton de um ângulo de 45º para fótons de:

    a) Raios-X com comprimento de onda de 0,025 nm.
    
    b) Raios-$\gamma$ com uma energia de 1MeV. 

Resposta:
Pelo princípir da conservação de energia temos sabemos que a energia inicial do fóton ($E_i$) é igual a energia final do fóton ($E_f$) mais a energia cinética ($K$), então temos:

\begin{equation}
    { E_i = E_f + K}
\end{equation}
Energia pode ser escrita em função da velocidade da luz($c$), do comprimento de onda($lambda$) e da constante de plank($h$):
\begin{equation}
    { E = \frac{hc}{\lambda}}
\end{equation}
Utilizando a equação X na equação Y, temos:
\begin{equation}
    { \frac{hc}{\lambda} = \frac{hc}{\lambda'} + K}
\end{equation}
A variação do comprimento de onda é:
\begin{equation}
    { \Delta\lambda = \lambda' - \lambda}
\end{equation}
Utilizando as duas equações acima temos:
\begin{equation}
    { \frac{hc}{\lambda} = \frac{hc}{(\Delta\lambda + \lambda}) + K}
\end{equation}

A energia cinética significa exatamente a diferença entre as energias iniciais e finais, diferença "perdida" em forma de energia cinética, então:
\begin{equation}
    { K =  \frac{hc}{\lambda} - \frac{hc}{(\Delta\lambda + \lambda)} = \frac{\Delta\lambda hc}{\lambda(\Delta\lambda + \lambda)}}
\end{equation}

a) $\lambda = 0,025.10^9m$,$\Delta\lambda = 7,11.10^{-13m}$
\begin{equation}
    {\frac{\Delta E}{E} =  \frac{7,11.10^{-13}}{(7,11.10^{-13}  + 0,025.10^9)} = 0,027 = 2,7 \% }
\end{equation}

Portanto a energia diminuiu em 2,7 \%.

b)
\begin{equation}
    {\frac {\Delta E}{E} = \frac{7,11.10^{-13}}{(7,11.10^{-13}  + 1,24.10^{-12})} = 0,36 = 36 \% }
\end{equation}
Portanto o fóton perdeu 36 \% da sua energia total.

\section {Questão 5}
a) 
\begin{equation}
    {V_corte = V = 0} 
\end{equation}
\begin{equation}
    {\frac{hv}{e} = \frac{W_0}{R}} 
\end{equation}
onde:
\begin{equation}
    {W_0 = h v_c} 
\end{equation}
\begin{equation}
    {v_c = \frach{W_0}{h} = 4,61.10^{14}Hz} 
\end{equation}
b)
\begin{equation}
    {V = \frac{h}{e}V - \frac{W_0}{e}} 
\end{equation}
\begin{equation}
    {W_0 = e(\frac{h}{e}v - V) = 1,6.10^{-19}(\frac{6,6.10^{-34}}{1,6.10^{-19}}.4,2.10^{19}) = 1,90 eV = 3,04 .10^{-19}J} 
\end{equation}

c) 
\begin{equation}
    {m  = \frac{Y-Y_0}{X-X_0} = 4,12.10^{-15}}
\end{equation}
\begin{equation}
    {h  =m.e =   6,6.10^{-14}J.s}
\end{equation}
\begin{figure} 
\includeimage[width=10cm]{A.jpg}
\end{figure}
\end{document}
